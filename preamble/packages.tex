
%  ╔══════════════════════════════════════════════════════════╗
%  ║                         packages                         ║
%  ╚══════════════════════════════════════════════════════════╝

\usepackage{inputenc}         % Used to compile any UTF-8 Character (and supports Unicode)
\usepackage{extarrows}         % Used to compile any UTF-8 Character (and supports Unicode)
\usepackage{stackengine}
\usepackage{scalerel}
\usepackage{amsmath, amssymb} % Used to write better mathematical expressions (ex. \mathbb{R})
\usepackage{stackrel}         % for left-right arrows, staking symbol above and bellow
\usepackage{mathrsfs}
\usepackage{euscript}
\usepackage[font=itshape]{quoting}
\usepackage{mathtools}
	\usepackage{enumitem}
\usepackage{amsthm}           % Used to create theorem enviroments.
\usepackage{graphicx}         % Let's you use images in LaTeX; ex the \includegraphics command.
\usepackage{hhline}
%\usepackage{luatexja}
%\usepackage{fontspec}
\usepackage{dsfont}
\usepackage{wasysym}          % emoji's!
\usepackage{float}            % package with ``H'' for figures and tables, ex. Images, tables, etc
%\usepackage{subcaption}      % More options with sub-captions.
\usepackage{setspace}         % More flexibility on spacing in document.
\usepackage{hyperref}         % let's me put hyperlinks
%\usepackage{tikz}             % for commutative diagrams https://tex.stackexchange.com/questions/419221/how-to-do-the-pushout-with-universal-property
\usepackage{tikz-cd}          % for commutative diagrams https://tex.stackexchange.com/questions/419221/how-to-do-the-pushout-with-universal-property
%\usepackage{quiver}           % for better commutative diagrams
\usepackage{longtable}        % create tables that extend past one page
%\usepackage{siunitx}         % required for alignment
%\usepackage{multirow}        % for multiple rows
\usepackage{microtype}        % makes nice text. Fixes some under-/over- filled box problems.
%\usepackage{fancyvrb}        % Let's you write code (different style than listings).
\usepackage{tcolorbox} 	      % Makes nice boxes for thms, cor, prop, or anything else.
%\usepackage{enumitem}        % More flexibility on enumeration (don't know much).
\usepackage{xcolor}           % Colour text.
\usepackage{wrapfig}          % To wrap text around figure
\usepackage{listings}         % Create nice colour code blocks (customized in preamble)
\usepackage{thmtools}         % Customize theorem environment (in preamble).
%\usepackage{marvosym}        % Provide ``basic'' symbols (hazard, religious, smiley face etc.)
\usepackage{array}            % \begin{table}{>{\em}c c} -> 1st column italic (bfseries for bold)
%\usepackage[english]{babel}  % If I want to blindtext in english.
\usepackage{blindtext}
\usepackage[a4paper, total={6in, 8in}]{geometry}
%\usepackage[symbol]{footmisc} % For daggers and asterisk for footnotes. 1: *, 2: dagger, etc.
\usepackage{fancyhdr}
\usepackage{titlesec}
%\usepackage{pst-plot}          %To make nice plots: https://tex.stackexchange.com/questions/47594/how-can-i-make-a-graph-of-a-function#47596
\usepackage[answerdelayed]{exercise}
\usepackage{etoolbox}          % for Dynkin diagrams
\usepackage{dynkin-diagrams}   % for Dynkin diagrams

% \usepackage{CJKutf8} % for some chinese

% to import figures via: https://github.com/gillescastel/inkscape-figures
\usepackage{import}
\usepackage{pdfpages}
\usepackage{transparent}

% This command is the ONLY command imported via the packages snippet, think of it as a tiny package written out
\newcommand{\incfig}[2][1]{%
    \def\svgwidth{#1\columnwidth}
    \import{./figures/}{#2.pdf_tex}
}

\pdfsuppresswarningpagegroup=1
