%header information
\pagestyle{fancy}
\setlength\headheight{20pt}
\renewcommand{\sectionmark}[1]{\markright{\thesection.\ #1}}
\renewcommand{\chaptermark}[1]{\markboth{#1}{}}
\fancyfoot[C,CO]{\textcolor{gray}{Nathanael Chwojko-Srawley}}
\fancyfoot[R,RO]{\thepage}{}

%chapter header format
\titleformat
{\chapter} % command
[display] % shape
{\bfseries\Huge\itshape} % format
{\thechapter} % label
{0.5ex} % sep
{
    \rule{\textwidth}{1pt}
    \vspace{1ex}
    \centering
} % before-code
[
\vspace{-0.5ex}%
\rule{\textwidth}{0.3pt}
] % after-code



%\ExerciseLevelInToc
\counterwithin{Exercise}{section}
\counterwithin{Answer}{section}
%\renewcommand{\ExerciseHeader}{\noindent\def\stackalignment{c}% code from https://tex.stackexchange.com/a/195118/101651
    %\stackunder{{\textbf{\large\ExerciseName}}}{\textcolor{black}{\rule{40pt}{4pt}}}\medskip}



%theorem code
\definecolor{dkgreen}{rgb}{0,0.6,0}
\definecolor{gray}{rgb}{0.5,0.5,0.5}
\definecolor{mauve}{rgb}{0.58,0,0.82}
\definecolor{lightyellow}{rgb}{1,1,0.6}
\definecolor{reddish}{HTML}{A01A3A}
%Theorems, propositions, Corollaries, Definitions


%Wanna implement this, but not working: https://tex.stackexchange.com/questions/547979/problem-numbering-environment
\tcbuselibrary{theorems}
\newtcbtheorem[number within=section]{thm}{Theorem}%
{colback=green!5,colframe=green!35!black,fonttitle=\bfseries}{th}
\newtcbtheorem[number within=section]{lem}{Lemma}%
{colback=blue!5,colframe=black!35!black,fonttitle=\bfseries}{lm}
\newtcbtheorem[number within=section]{prop}{Proposition}%
{colback=white!5,colframe=white!35!black,fonttitle=\bfseries}{pr}
\newtcbtheorem[number within=section]{cor}{Corollary}%
{colback=blue!5,colframe=blue!35!black,fonttitle=\bfseries}{co}
\newtcbtheorem[number within=section]{axiom}{Axiom}%
{colback=black!5,colframe=yellow!35!black,fonttitle=\bfseries}{ax}
\newtcbtheorem[number within=section]{defn}{Definition}%
{colback=red!5,colframe=red!35!black,fonttitle=\bfseries}{df}



\declaretheoremstyle[
spaceabove=6pt, spacebelow=6pt,
headfont=\normalfont\bfseries,
notefont=\mdseries, notebraces={(}{)},
bodyfont=\normalfont,
postheadspace=1em,
numberwithin=section
]{exstyle}

\declaretheoremstyle[
spaceabove=6pt, spacebelow=6pt,
headfont=\normalfont\bfseries,
notefont=\mdseries, notebraces={(}{)},
bodyfont=\normalfont,
postheadspace=1em,
headpunct={},
qed=$\blacktriangledown$,
numbered=no
]{solstyle}

\declaretheoremstyle[
spaceabove=6pt, spacebelow=6pt,
headfont=\normalfont\bfseries,
notefont=\mdseries, notebraces={(}{)},
bodyfont=\normalfont,
postheadspace=1em,
headpunct={},
numbered=no
]{questyle}

\declaretheorem[style=solstyle]{solution}
\declaretheorem[style=questyle]{question}
%\declaretheorem[style=solstyle]{remark} (got a ``Note'' environment at the bottom``)
%\declaretheorem[style=solstyle]{intuition}


%shortcuts for this file (newcommand and renewcommand)


% Custom Commands
\newcommand{\A}{\mathbb{A}}
\newcommand{\F}{\mathbb{F}}
\newcommand{\N}{\mathbb{N}}
\newcommand{\Q}{\mathbb{Q}}
\newcommand{\R}{\mathbb{R}}
\newcommand{\V}{\mathbb{V}}
\newcommand{\Z}{\mathbb{Z}}
\newcommand{\C}{\mathbb{C}}
\renewcommand{\H}{\mathbb{H}}

\newcommand{\CA}{\mathcal{A}}
\newcommand{\CB}{\mathcal{B}}
\newcommand{\CC}{\mathcal{C}}
\newcommand{\CD}{\mathcal{D}}
\newcommand{\CF}{\mathcal{F}}
\newcommand{\CG}{\mathcal{G}}
\newcommand{\CH}{\mathcal{H}}
\newcommand{\CI}{\mathcal{I}}
\newcommand{\CL}{\mathcal{L}}
\newcommand{\CN}{\mathcal{N}}
\newcommand{\CO}{\mathcal{O}}
\newcommand{\CP}{\mathbb{C}\mathrm{P}}
\newcommand{\CQ}{\mathcal{Q}}
\newcommand{\CR}{\mathcal{R}}
\newcommand{\CK}{\mathcal{K}}
\newcommand{\CS}{\mathcal{S}}
\newcommand{\CT}{\mathcal{T}}
\newcommand{\CV}{\mathcal{V}}
\newcommand{\CZ}{\mathcal{Z}}
\newcommand{\ff}{\mathscr{F}}
\newcommand{\EF}{\EuScript{F}}
\newcommand{\EL}{\EuScript{L}}
\newcommand{\EO}{\EuScript{O}}
\newcommand{\EC}{\EuScript{C}}
\newcommand{\co}{\mathfrak{o}} %mathcal{o} looks like wreath product, so I changed it to mathfrak
\newcommand{\fa}{\mathfrak{a}}
\newcommand{\fb}{\mathfrak{b}}
\newcommand{\fc}{\mathfrak{c}}
\newcommand{\fd}{\mathfrak{d}}
\newcommand{\fm}{\mathfrak{m}}
\newcommand{\fn}{\mathfrak{n}}
\newcommand{\fo}{\mathfrak{o}}
\newcommand{\fp}{\mathfrak{p}}
\newcommand{\fq}{\mathfrak{q}}
\newcommand{\FA}{\mathfrak{A}}
\newcommand{\FB}{\mathfrak{B}}
\newcommand{\FP}{\mathfrak{P}}
\newcommand{\FC}{\mathfrak{C}}
\newcommand{\FD}{\mathfrak{D}}
\newcommand{\FM}{\mathfrak{M}}
\newcommand{\FN}{\mathfrak{N}}
\newcommand{\FO}{\mathfrak{O}}

\newcommand{\sse}{\subseteq}
\newcommand{\nsse}{\not\subseteq}
\newcommand{\ssne}{\subsetneq}
\newcommand{\qn}{\quad\newline}
\newcommand{\placeholder}{\_\_\_\_\_}
\newcommand{\subg}{\leqslant}
%\newcommand{\subgne}{\lneqslant}
\newcommand{\supg}{\geqslant}
\newcommand{\nsubg}{\trianglelefteq}
\newcommand{\csubg}{\blacktriangleleft}
\newcommand{\nsupg}{\trianglerighteq}
\newcommand{\isosubg}{\lesssim}
\renewcommand{\mid}{\big|}
\renewcommand{\phi}{\varphi}

\newcommand{\oln}{\overline}
\renewcommand{\bf}[1]{\textbf{#1}}

\newcommand{\ev}{\operatorname{ev}}
\newcommand{\tr}{\operatorname{tr}}
\newcommand{\nm}{\operatorname{Nm}}
\newcommand{\op}{\operatorname{op}}
\newcommand{\ad}{\operatorname{ad}}
\newcommand{\im}{\operatorname{im}}
\newcommand{\id}{\operatorname{id}}
\newcommand{\ch}{\operatorname{ch}}
\newcommand{\ab}{\operatorname{ab}}
\newcommand{\SL}{\operatorname{SL}}
\newcommand{\GL}{\operatorname{GL}}
\newcommand{\rad}{\operatorname{rad}}
\newcommand{\vol}{\operatorname{vol}}
\newcommand{\Gal}{\operatorname{Gal}}
\newcommand{\Rep}{\operatorname{Rep}}
\newcommand{\Aut}{\operatorname{Aut}}
\newcommand{\Sym}{\operatorname{Sym}}
\newcommand{\Mod}{\operatorname{Mod}}
\newcommand{\Grp}{\operatorname{Grp}}
\newcommand{\Hom}{\operatorname{Hom}}
\newcommand{\Emb}{\operatorname{Emb}}
\newcommand{\Ext}{\operatorname{Ext}}
\newcommand{\End}{\operatorname{End}}
\newcommand{\Inn}{\operatorname{Inn}}
\newcommand{\Ind}{\operatorname{Ind}}
\newcommand{\Out}{\operatorname{Out}}
\newcommand{\Jac}{\operatorname{Jac}}
\newcommand{\Nil}{\operatorname{Nil}}
\newcommand{\Tor}{\operatorname{Tor}}
\newcommand{\Ann}{\operatorname{Ann}}
\newcommand{\Syl}{\operatorname{Syl}}
\newcommand{\Res}{\operatorname{Res}}
\newcommand{\lcm}{\operatorname{lcm}}
\newcommand{\conj}{\operatorname{conj}}
\newcommand{\disc}{\operatorname{disc}}
\newcommand{\stab}{\operatorname{stab}}
\newcommand{\kdim}{\operatorname{kdim}}
\newcommand{\coker}{\operatorname{coker}}
\newcommand{\Obj}{\operatorname{Obj}}
\newcommand{\Mor}[1]{\operatorname{Mor}(\textbf{#1})}
\newcommand{\Frac}{\operatorname{Frac}}
\newcommand{\Frob}{\operatorname{Frob}}
\newcommand{\Spec}{\operatorname{Spec}}
\newcommand{\mSpec}{\operatorname{mSpec}}
\newcommand{\Tens}[1]{#1\otimes \_\_}
\newcommand{\Char}{\operatorname{char}}
\newcommand{\Span}{\operatorname{span}}
\newcommand{\Lim}{\varprojlim}
\newcommand{\colim}{\varinjlim}
\newcommand{\coLim}{\varinjlim}
\newcommand{\Dlim}{\lim\limits_{\longrightarrow}}
\newcommand{\coDlim}{\lim\limits_{\longleftarrow}}
\newcommand{\fHom}[1]{\operatorname{Hom}(#1, \_\_)}
\newcommand{\cofHom}[1]{\operatorname{Hom}(\_\_, #1)}
\newcommand{\trdeg}{\operatorname{trdeg}}
\newcommand{\acton}{\curvearrowright}
\newcommand{\actson}{\curvearrowright}

\renewcommand{\Re}{\operatorname{Re}}
\renewcommand{\Im}{\operatorname{Im}}

%for saturated ideals in the localization section
\newcommand{\LIm}{{}^{\iota} }
\newcommand{\LPre}{{}^{\pi} }

% Arrows
\newcommand{\rw}{\rightarrow}
\newcommand{\rrw}{\rightrightarrows}
\newcommand{\Rw}{\Rightarrow}
\newcommand{\lw}{\leftarrow}
\newcommand{\Lw}{\Leftarrow}
\newcommand{\lrw}{\leftrightarrow}
\newcommand{\LRw}{\Leftrightarrow}

\newcommand{\dcup}{\sqcup}
\newcommand{\Dcup}{\bigsqcup}
\newcommand{\set}[2]{\left\{#1 \ \middle|\ #2\right\}}

\newcommand\mapsfrom{\mathrel{\reflectbox{\ensuremath{\mapsto}}}}

\newcommand{\tikzcircle}[2][red,fill=black]{\tikz[baseline=-0.5ex]\draw[#1,radius=#2] (0,0) circle ;}%

\def\rddots{\mathstrut^{.^{.^{.}}}}

\makeatletter
\newcommand*\bigcdot{\mathpalette\bigcdot@{.5}}
\newcommand*\bigcdot@[2]{\mathbin{\vcenter{\hbox{\scalebox{#2}{$\m@th#1\bullet$}}}}}
\makeatother

%% new delimiter
\DeclarePairedDelimiter{\ceil}{\lceil}{\rceil}


\setlength{\parindent}{0pt} % don't like indentation infront of paragraphs
\setlength{\parskip}{1ex}   %so that there is still space between paragarphs
%\declaretheorem[style=exstyle]{example}

\tcbuselibrary{most}

%Custom enviroments
 \newenvironment{note}
 {\begin{tcolorbox}[enhanced, sharp corners, colback=white , borderline={0.2pt}{0pt}{black}]
   \begin{quote}\textbf{Note}:
   }{
   \end{quote}\end{tcolorbox}}

 \newenvironment{intuition}
 {\begin{quote}\textbf{Intuition}:
   }{
   \end{quote}}

\newenvironment{titledBox}[1]
 {\begin{tcolorbox}[enhanced, sharp corners, colback=white , borderline={0.2pt}{0pt}{black}]
   \begin{quote}\textbf{#1}
   }{
   \end{quote}\end{tcolorbox}}

\def\exampletext{Example} % If English
\colorlet{colexam}{red!55!black} % Global example color


\newtcbtheorem[auto counter, number within=chapter]{example}{Example}{% \newtcolorbox[use counter=testexample]{testexamplebox}{%
% Example Frame Start
empty,% Empty previously set parameters
title={\exampletext \thetcbcounter #1},% use \thetcbcounter to access the testexample counter text
% Attaching a box requires an overlay
attach boxed title to top left,
   % Ensures proper line breaking in longer titles
   minipage boxed title,
% (boxed title style requires an overlay)
boxed title style={empty,size=minimal,toprule=0pt,top=4pt,left=3mm,overlay={}},
coltitle=colexam,fonttitle=\bfseries,
before=\par\medskip\noindent,parbox=false,boxsep=0pt,left=3mm,right=0mm,top=2pt,breakable,pad at break=0mm,
   before upper=\csname @totalleftmargin\endcsname0pt, % Use instead of parbox=true. This ensures parskip is inherited by box.
% Handles box when it exists on one page only
overlay unbroken={\draw[colexam,line width=.5pt] ([xshift=-0pt]title.north west) -- ([xshift=-0pt]frame.south west); },
% Handles multipage box: first page
overlay first={\draw[colexam,line width=.5pt] ([xshift=-0pt]title.north west) -- ([xshift=-0pt]frame.south west); },
% Handles multipage box: middle page
overlay middle={\draw[colexam,line width=.5pt] ([xshift=-0pt]frame.north west) -- ([xshift=-0pt]frame.south west); },
% Handles multipage box: last page
overlay last={\draw[colexam,line width=.5pt] ([xshift=-0pt]frame.north west) -- ([xshift=-0pt]frame.south west); }%
}{ex}



\def\prooftext{Proof} % If English
\NewDocumentEnvironment{Proof}{ O{} }
{
	\colorlet{colexam}{black!55!black} % Global example color
	\newtcolorbox[number within=section]{testproofbox}{% \newtcolorbox[use counter=testexample]{testexamplebox}{%
		% Example Frame Start
		empty,% Empty previously set parameters
		title={\emph{\prooftext} \thetcbcounter: #1},% use \thetcbcounter to access the testexample counter text
		% Attaching a box requires an overlay
		attach boxed title to top left,
		   % Ensures proper line breaking in longer titles
		   minipage boxed title,
		% (boxed title style requires an overlay)
		boxed title style={empty,size=minimal,toprule=0pt,top=4pt,left=3mm,overlay={}},
		coltitle=colexam,fonttitle=\bfseries,
		before=\par\medskip\noindent,parbox=false,boxsep=0pt,left=3mm,right=0mm,top=2pt,breakable,pad at break=0mm,
		   before upper=\csname @totalleftmargin\endcsname0pt, % Use instead of parbox=true. This ensures parskip is inherited by box.
		% Handles box when it exists on one page only
		overlay unbroken={\draw[colexam,line width=.5pt] ([xshift=-0pt]title.north west) -- ([xshift=-0pt]frame.south west); },
		% Handles multipage box: first page
		overlay first={\draw[colexam,line width=.5pt] ([xshift=-0pt]title.north west) -- ([xshift=-0pt]frame.south west); },
		% Handles multipage box: middle page
		overlay middle={\draw[colexam,line width=.5pt] ([xshift=-0pt]frame.north west) -- ([xshift=-0pt]frame.south west); },
		% Handles multipage box: last page
		overlay last={\draw[colexam,line width=.5pt] ([xshift=-0pt]frame.north west) -- ([xshift=-0pt]frame.south west); },%
		}
	\begin{testproofbox}
}
{\end{testproofbox}\endlist}
%/home/nathanael/Documents/Books/Mathematics/Dummit and Foote - Abstract Algebra.pdf
%/home/nathanael/Documents/Books/Mathematics/


%for Dynkin diagrams
\def\row#1/#2!{#1_{\IfStrEq{#2}{}{n}{#2}} & \dynkin{#1}{#2}\\}
\newcommand{\tble}[1]{
   \renewcommand*\do[1]{\row##1!}
   \[
      \begin{array}{ll}\docsvlist{#1}\end{array}
   \]
}



\stackMath %to be able to stack text in math mode

% For devnagari font for that one example
\newcommand\dn{\catcode`\~=12
           \fontspec[Script=Devanagari,Mapping=velthuis-sanskrit]{Nakula}}
